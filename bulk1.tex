Sir Pelham Grenville Wodehouse, KBE, (/ˈwʊdhaʊs/; 15 October 1881 – 14
February 1975) was an English humorist whose body of work includes
novels, short stories, plays, poems, song lyrics and numerous pieces
of journalism. He enjoyed enormous popular success during a career
that lasted more than seventy years, and his many writings continue to
be widely read. Despite the political and social upheavals that
occurred during his life, much of which was spent in France and the
United States, Wodehouse's main canvas remained that of a pre- and
post-World War I English upper class society, reflecting his birth,
education and youthful writing career.

An acknowledged master of English prose, Wodehouse has been admired
both by contemporaries such as Hilaire Belloc, Evelyn Waugh and
Rudyard Kipling and by recent writers such as Christopher Hitchens,
Stephen Fry,[1] Douglas Adams,[2] J. K. Rowling,[3] and John Le
Carré.[4]

Best known today for the Jeeves and Blandings Castle novels and short
stories, Wodehouse was also a playwright and lyricist who was part
author and writer of 15 plays and of 250 lyrics for some 30 musical
comedies, many of them produced in collaboration with Jerome Kern and
Guy Bolton. He worked with Cole Porter on the musical Anything Goes
(1934), wrote the lyrics for the hit song "Bill" in Kern's Show Boat
(1927), wrote lyrics to Sigmund Romberg's music for the Gershwin –
Romberg musical Rosalie (1928) and collaborated with Rudolf Friml on a
musical version of The Three Musketeers (1928). He is in the
Songwriters Hall of Fame.[5]

Wodehouse spent the last decades of his life in the United States,
becoming an American citizen in 1955, because of controversy that
arose after he made five radio broadcasts from Germany during World
War II, where he had been interned by the Germans for a
year. Speculation after the broadcasts led to allegations of
collaboration and treason. Some libraries banned his books. Although
an MI5 investigation later cleared him of any such crimes, he never
returned to England.


